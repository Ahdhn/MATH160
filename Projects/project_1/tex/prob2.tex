\section*{Problem No.2} \label{sec:prob2}
Using the plurality vote method, we rank each candidate by the amount of first place votes they receive. The winner of the election is Bernie Sanders with $96$ first place votes.\vspace{1pc}

\noindent Using the average rank method, we rank each candidate by assigning points from $1$ to $n$, where $1$ is the voter's least favorite candidate and $n$ the voter's most favorite candidate, where $n$ is the total number of candidates. After collecting all the data from the all the voters, we take the sum of all the points for each candidate and divide by the total number of voters to get the average rank of each candidate. By this method, Hilary Clinton wins the election with a rank of $3.8583$, slightly above Bernie Sanders who has a rank of $3.7417$.\vspace{1pc}

\noindent Using the Borda count method, we rank each candidate by assigning points from $0$ to $n-1$, where $0$ is the voter's least favorite candidate and $n-1$ is the voter's most favorite candidate, where $n$ is the total number of candidates. After collecting all the data from the voters, we take the sum of all the points for each candidate, and the candidate with the highest sum (the Borda score) is the winner. By this method, Hilary Clinton is the winner with a Borda score of $686$, and Bernie Sanders is in second with a Borda score of $658$.\vspace{1pc}

\noindent Using the W-Borda count method, we rank each candidate just like how we do in the Borda count method. However, the assigned points are given by a vector W, where $w_1 \geq w_2 \geq ... \geq w_n$, with $w_1$ being the points assigned to the candidate in first place and $w_n$ to the candidate in last place. Using the W vector values $[1\hspace{1mm} 1/2\hspace{1mm} 1/3\hspace{1mm} 1/4\hspace{1mm} 1/5]$ we can see that Bernie Sanders wins the election with a score of $148.5$ and Hilary Clinton with a score of $147.9$. However, if we change the W vector values to $5\hspace{1mm} 4\hspace{1mm} 3\hspace{1mm} 2\hspace{1mm} 1]$, Hilary Clinton wins the election with a score of $926$ and Bernie Sanders having a score of $898$.\vspace{1pc}

\noindent Using the Pagerank algorithm, we rank each candidate by calculating the A matrix. We do this by creating a $5x5$ matrix for each voter denoting their possible choices, and mark each candidate with a value of $1$ if the candidate on the given row loses to the corresponding candidate on the given column. By taking the summation of all the voters' matrices, and dividing each column by the respective column sum, we get the A matrix. We then calculate $\tilde A$ using the equation given in class, and from there we compute $Ar=r$ using the power method. By this method, we get that Hilary Clinton wins the election with a score of $0.24981$ while Bernie Sanders has a score of $0.246$.\vspace{1pc}

\noindent We see that depending on the vote method chosen, we either see Hilary Clinton or Bernie Sanders as the winner. By plurality method, we see that Bernie Sanders is the winner because he has the most first place votes. However, by the average rank method, Borda count method, and Page Rank method, we see that Hilary Clinton is the winner. This tells us that to a portion of the voters, Bernie Sanders was the top choice, and to the other portion, Bernie Sanders was more towards the less favorable side, while Hilary Clinton had most of her votes towards the favorable side.\vspace{1pc}

\noindent Using the W Borda method, we see that manipulating the W vector can give us two different winners, either Hilary Clinton or Bernie Sanders. A vector with a larger point margin between the first and last candidates usually results in Hilary Clinton winning, while a vector with a smaller point margin between the first and last candidates usually results in Bernie Sanders winning.\vspace{1pc}

\noindent In our opinion, Page Rank is the fairest method to count votes because the margin in between each candidate is the smallest. This will give us the best representation of how close the election is, and if there is a need for a recount.\vspace{1pc}

\begin{figure}[tbh]
 \centering    
\begin{tabular}{ |p{2cm}|p{2.5cm}|p{3cm}|p{2cm}|p{2.5cm}|p{2.5cm}|}
 \hline
  Plurality & Average &  Borda & W-Borda1 & W-Borda2 & Page Rank \\ \hhline{|=|=|=|=|=|=|}
 \hline
   BS (96) & HC (3.8583) & HC (686) & HC (926) & BS (148.05) & HC (0.24981) \\
   HC (85) & BS (3.7417) & BS (658) & BS (898) & HC (147.9)  & BS (0.246)   \\
   DT (44) & DT (2.6958) & DT (407) & DT (647) & DT (101.15) & DT (0.19265) \\
   TC (8)  & JK (2.5375) & JK (369) & JK (609) & JK (78.1167)& JK (0.16525) \\
   JK (7 ) & TC (2.1667) & TC (280) & TC (520) & TC (72.7833)& TC (0.14629) \\ 
 \hline
\end{tabular}
\caption{Ranking of the candidates using different methods along with the each method rating between parenthesis. W-Borda1 corresponds to the weight vector of $[5\ 4\ 3\ 2\ 1]$ while $W-Borda2$ is for weight vector  $[1\ \frac{1}{2}\ \frac{1}{3}\ \frac{1}{4}\ \frac{1}{5}]$.} 
\end{figure} 
    
    
    
    
    
    
    
    
    
    
    
    
    
    
    
    
    
    
    
    
    