\section*{Problem No.2} \label{sec:prob2}
Using the plurality vote method, we rank each candidate by the amount of first place votes they receive. The winner of the election is Bernie Sanders with $96$ first place votes.\vspace{1pc}

\noindent Using the average rank method, we rank each candidate by assigning points from $1$ to $n$, where $1$ is the voter's least favorite candidate and $n$ the voter's most favorite candidate, where $n$ is the total number of candidates. After collecting all the data from the all the voters, we take the sum of all the points for each candidate and divide by the total number of voters to get the average rank of each candidate. By this method, Hilary Clinton wins the election with a rank of $3.8583$, slightly above Bernie Sanders who has a rank of $3.7417$.\vspace{1pc}

\noindent Using the Borda count method, we rank each candidate by assigning points from $0$ to $n-1$, where $0$ is the voter's least favorite candidate and $n-1$ is the voter's most favorite candidate, where $n$ is the total number of candidates. After collecting all the data from the voters, we take the sum of all the points for each candidate, and the candidate with the highest sum (the Borda score) is the winner. By this method, Hilary Clinton is the winner with a Borda score of $686$, and Bernie Sanders is in second with a Borda score of $658$.\vspace{1pc}

\noindent Using the W-Borda count method, we rank each candidate just like how we do in the Borda count method. However, the assigned points are given by a vector W, where $w_1 \geq w_2 \geq ... \geq w_n$, with $w_1$ being the points assigned to the candidate in first place and $w_n$ to the candidate in last place. Using the W vector values $[10\hspace{1mm} 4\hspace{1mm} 3\hspace{1mm} 2\hspace{1mm} 1]^T$ we can see that Bernie Sanders wins the election with a score of $1378$ and Hilary Clinton with a score of $1351$. However, if we change the W vector values to $10\hspace{1mm} 9\hspace{1mm} 3\hspace{1mm} 2\hspace{1mm} 1]^T$, Hilary Clinton wins the election with a score of $1811$ and Bernie Sanders having a score of $1653$. If we change the W vector values to $[0\hspace{1mm} 0\hspace{1mm} 0\hspace{1mm} 0\hspace{1mm} 0]^T$ we will get a five way tie.\vspace{1pc}

\noindent Using the Pagerank algorithm, we rank each candidate by looking at each voter's 

\begin{figure}[tbh]
 \centering    
\begin{tabular}{ |p{2cm}|p{2.5cm}|p{3cm}|p{2cm}|p{2.5cm}|p{2.5cm}|}
 \hline
  Plurality & Average &  Borda & W-Borda1 & W-Borda2 & Page Rank \\ \hhline{|=|=|=|=|=|=|}
 \hline
   BS (96) & HC (3.8583) & HC (686) & HC (926) & BS (148.05) & HC (0.24981) \\
   HC (85) & BS (3.7417) & BS (658) & BS (898) & HC (147.9)  & BS (0.246)   \\
   DT (44) & DT (2.6958) & DT (407) & DT (647) & DT (101.15) & DT (0.19265) \\
   TC (8)  & JK (2.5375) & JK (369) & JK (609) & JK (78.1167)& JK (0.16525) \\
   JK (7 ) & TC (2.1667) & TC (280) & TC (520) & TC (72.7833)& TC (0.14629) \\ 
 \hline
\end{tabular}
\caption{Ranking of the candidates using different methods along with the each method rating between parenthesis. W-Borda1 corresponds to the weight vector of $[5\ 4\ 3\ 2\ 1]$ while $W-Borda2$ is for weight vector  $[1\ \frac{1}{2}\ \frac{1}{3}\ \frac{1}{4}\ \frac{1}{5}]$.} 
\end{figure} 
    
    
    
    
    
    
    
    
    
    
    
    
    
    
    
    
    
    
    
    
    