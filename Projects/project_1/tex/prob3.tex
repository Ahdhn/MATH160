\section*{Problem No.3} \label{sec:prob3}

\textbf{(i)} Theory of Singular value decomposition allows us to decompose the matrix G as a sum of many rank-one matrices. Rank of a matrix is defined as the maximum number of linearly independent column vectors. Using this fact, we can give a ranking because there's only one linearly independent column vector which implies that it will scale the values of the difficulty score the students gave for each questions. Thus, The lowest value of the rating will be the most difficult.
\\
\\
\textbf{(ii)} 
Massey least square method calculates the sum of score differential between two teams in this case two questions. It output the ratings/difficulty of each question by approximately solving the system of linear equality (the score difference) between all of the questions. The output with the greatest value would be the would be considered easiest. 
\\
\\
\textbf{(iii)} The Result
\\
The rating of the difficulty of each equation in the test using SVD method, Massey's network method and Colley's method showed similar but not exact ranking system of the difficulty of the questions. The values were different but the rankings were the same between the results from Colley's method and Massey's method. Furthermore, all three methods ranked question 7 as the hardest, question 5 as third easiest and question 1 as the easiest question. The placement for 2nd, 4th, 5th and 6th easiest questions were different in result from SVD compared to Massey's. Massey had the greatest range of the scores.
